\subsection{Nonstructural Empirical Evidence}\label{subsec:microEvidence}


\subsubsection{Background}

Above we cite efforts to calibrate structural parameters of specific epidemiological models to data.  %But our definition of an epidemiological process as one in which social interactions affect people's beliefs and consequent economic behaviors means .
Here, we summarize literature that collects evidence in ways not targeted to estimating parameters, but that may nevertheless be useful in guiding the construction of epidemiological models.  In principle, such work could help answer questions like
\begin{quote}
    \normalfont
    \begin{verbatimwrite}{./Slides/MicroEvidence}
\begin{enumerate}
	\item When do socially transmitted beliefs cause consequential economic decisions?
    \item What are characteristics of sources and recipients of expectational infection?
    \item Through which media are expectations mostly transmitted?
    \item What kinds of information/expectations are more infectious?
    \item How can \cite{manski1993identification}'s reflection problem be addressed?
    \end{enumerate}
\end{verbatimwrite}
\begin{enumerate}
	\item When do socially transmitted beliefs cause consequential economic decisions?
    \item What are characteristics of sources and recipients of expectational infection?
    \item Through which media are expectations mostly transmitted?
    \item What kinds of information/expectations are more infectious?
    \item How can \cite{manski1993identification}'s reflection problem be addressed?
    \end{enumerate}

\end{quote}

Among the reasons epidemiological modeling has been slow to spread, one is surely that every one of these questions has been difficult to answer using traditional data sources.  But new data, particularly the burgeoning ``social network'' data, offer rich opportunities for profoundly improving our ability to answer these questions.

\subsubsection{Papers Using Proxies for Social Connections}

In the absence of direct evidence about the nature and frequency of social contacts between people, the economics literature has naturally relied upon plausible proxies.   For instance,  \href{http://www.columbia.edu/~hh2679/ThyNeighborJF.pdf}{\cite{hong2005thy}} found that fund managers tend to buy similar stocks to other fund managers in the same city. \href{https://github.com/iworld1991/EpiExp/blob/master/Literature/hvide2015social.pdf}{\cite{hvide2015social}} found that stock market investment decisions of individuals are positively correlated with those of coworkers.  \href{https://www.jstor.org/stable/10.1086/592415}{\cite{cohen2008small}} shows that fund managers place larger bets (that perform better) on firms to whose employees they are socially connected.  In addition, social interaction also affects stock market participation and stock choices, as shown in  \href{https://github.com/iworld1991/EpiExp/blob/master/Literature/hong2004social.pdf}{\cite{hong2004social}}, \href{https://onlinelibrary.wiley.com/doi/abs/10.1111/j.1540-6261.2008.01364.x}{\cite{brown2008neighbors}}, and \href{https://github.com/iworld1991/EpiExp/blob/master/Literature/ivkovic2007information.pdf}{\cite{ivkovic2007information}}. % hong2004social:  more social households more likely to invest in the stock market using data from HRS;  brown2008neighbors:  one more likely owns stocks in higher ownership communities, instrumenting the community ownership by nonnative residents' ownership.

In the context of housing market investment, one paper that explicitly emphasizes the transmission of information or beliefs by social contacts, and specifically mentions epidemiological mechanisms as potential channels of transmission, is
 \href{https://www.aeaweb.org/articles?id=10.1257/aer.20171611&from=f}{\cite{bayer2021speculative}}, which shows that novice investors were more likely to enter the market (in speculative ways) after seeing that their immediate neighbors had invested.


A small literature provides direct evidence for social dynamics during bank run episodes, as we described in section~\ref{subsec:Contagion}.  \href{https://www.aeaweb.org/articles?id=10.1257/aer.102.4.1414}{\cite{iyer2012understanding}} study the dynamics of an actual bank run using high-frequency data on deposit withdrawals among persons connected in a social network.   \href{https://www.aeaweb.org/articles?id=10.1257/aer.90.5.1110}{\cite{kelly2000market}} showed that depositors who learned bad news about a bank from acquaintances were the first to close their accounts.

There is a also large literature that finds evidence for `peer effects' on people's financial choices; a natural interpretation of such effects is that in many cases they likely reflect epidemiological transmission of beliefs, but much of this literature has been content to document the existence of the effect while remaining mute on the mechanism.  (See \cite{kuchler2021social} for a comprehensive survey of the substantial literature on peer effects on financial behaviors).


% frm \cite{galesic2018asking}:  Abstract: Abstract Election outcomes can be difficult to predict. A recent example is the 2016 US presidential election, in which Hillary Clinton lost five states that had been predicted to go for her, and with them the White House. Most election polls ask people about their own voting intentions: whether they will vote and, if so, for which candidate. We show that, compared with own-intention questions, social-circle questions that ask participants about the voting intentions of their social contacts improved predictions of voting in the 2016 US and 2017 French presidential elections. Responses to social-circle questions predicted election outcomes on national, state and individual levels, helped to explain last-minute changes in people’s voting intentions and provided information about the dynamics of echo chambers among supporters of different candidates.

\subsubsection{Directly Measured Social Networks}

In a world with ubiquitous social networks, the set of people who can influence economic expectations is not limited to peers who are physically nearby.  \href{https://www.journals.uchicago.edu/doi/abs/10.1086/700073}{\cite{bailey2018economic,bailey2019house}} show, essentially, that people who happen randomly to have social-network friends in distant cities where home prices have increased are more optimistic about their local housing market, and more likely to buy, than people whose remote friends happen to live in places where house prices declined.\footnote{See \kpshousingexpectationFull, for a discussion of various drivers of housing price expectations.}  Using Facebook data, \cite{makridis2020learning} find that during the COVID-19 crisis, the severity of the decline in consumption in a county was partly explained by the severity of the epidemic in the locales to which that county had especially dense social ties -- even when those locales were geographically distant.

%In comparison with earlier empirical work that identifies social interaction effects via proxies of local connections,  this new body of work benefits from the availability of large-scale nationwide online network data such as the social-connectedness-index (SCI) constructed using the universe of Facebook users. \cite{bailey_social_2018}  validates the representativeness of these measures and its correlation with important economic  and social activities.




%Word-of-mouth communications might be even particularly important in spreading the information in fraudulent contagion and speculative investment activities. For instance, \href{https://github.com/iworld1991/EpiExp/blob/master/Literature/rantala2019investment.pdf}{\cite{rantala2019investment}} provides direct evidence on diffusion of investment ideas among a large Ponzi scheme. \href{https://files.fisher.osu.edu/department-finance/public/information_networks_evidence_from_illegal_insider_trading_tips.pdf}{\cite{ahern2017information}}shed light on the information flow of illegal insider trading among strong social ties.

\subsubsection{News Media}

Social communication not only takes the form of conversations within direct social circles but also via mass media.

\textbf{\textit{Financial Markets.}}  Rather than attempting a broad discussion of the diffuse literature on the relationship between the media and financial markets, we refer the reader to ``The Role of Media in Finance'' by~\cite{TETLOCK2015701}.  Here we highlight just a few contributions that are particularly noteworthy for our purposes. %\href{http://faculty.haas.berkeley.edu/odean/papers\%20current\%20versions/allthatglitters_rfs_2008.pdf}{\cite{tetlock_giving_2007}} attempted to systematically conduct `sentiment analysis' of news coverage and shows that what he characterizes as `non-fundamental' sentiment from financial news drives trading volumes of the relevant stocks.
\href{https://www.researchgate.net/publication/227465410_Journalists_and_the_Stock_Market}{\cite{dougal2012journalists}} attempt to measure the impact of the opinions of individual \textit{Wall Street Journal} columnists on market outcomes; this is a particularly clear example of a result with a straightforward interpretation using a `common source' epidemiological model.  \href{https://www.public.asu.edu/~dsosyura/ResearchPapers/Rumor\%20Has\%20It\%20--\%20Sensationalism\%20in\%20Financial\%20Media.pdf}{\cite{ahern2015rumor}} found that younger and more inexperienced journalists tended to write more sensational and ambiguous news reports about corporate mergers, so that the youth and inexperience of the journalist had predictive power for the market impact of merger stories. %\href{http://faculty.haas.berkeley.edu/odean/papers\%20current\%20versions/allthatglitters_rfs_2008.pdf}{\cite{barber_all_2008}} found that individual investors are more likely to buy stocks that are ``in the news'' or that have had extreme recent one-day returns.
\href{https://www.stern.nyu.edu/sites/default/files/assets/documents/con_040497.pdf}{\cite{soo_quantifying_2015}} used news sources to construct an index of ``animal spirits'' in the housing market and argued that this index had predictive power for housing prices.

\textbf{\textit{Macroeconomics}}.  There is a substantial literature on the nature of news media coverage of macroeconomic developments (mostly outside of economics, cf.~\cite{soroka2015s}; ~\cite{damstra2021economy}; see  \cite{bybee2020structure} for recent work by economists), but the slow-moving nature of macroeconomic outcomes makes it difficult to distinctly identify consequences of the nature of the coverage from the consequences of the economic events themselves.  \cite{nimark2014man} is nevertheless able to show that particularly surprising events seem to have identifiable macroeconomic consequences out of proportion to what might be judged to be their appropriate impact.\footnote{See also \cite{chahrour2021sectoral} provide evidence that coverage about newsworthy events that affect particular sectors but are unrepresentative of broader developments can affect broader hiring decisions.}

An indirect approach is to attempt to measure the effect of news coverage on consumer sentiment, and then to rely upon a separate literature that has found that consumer sentiment seems to have predictive power for economic outcomes (\cite{ludvigson2004consumer}, \cite{cfwSentiment}).  One example is a clever paper by \cite{doms2004consumer} showing that consumer sentiment is driven by news coverage even during periods in which coverage is inconsistent with economic conditions.\footnote{For further evidence that news coverage is a key source of people's views, see \cite{lamla2012role}, though see~\cite{pfajfar2013news} for a skeptical view.}

New ways of pursuing these kinds of ideas may be feasible using data like Google Trends search queries, which~\cite{choi2012predicting} have shown can predict sentiment data well and can serve as a real-time measure of the degree of internet users' interest in economic topics.

Perhaps the most notable recent work relating media to macroeconomics has been that of \cite{baker2016measuring}, who use media sources to construct an index of ``economic policy uncertainty'' and find that it has predictive power for macroeconomic outcomes beyond what can be extracted from the usual indicators.  The extent to which an epidemiological mechanism is necessary in making sense of this finding is unclear; the authors' own interpretation seems to be mainly that they are measuring a fundamental fact about the world (the policymaking process inherently and unavoidably generates uncertainty.)

It is possible, however, that the degree of uncertainty the authors measure is affected by the structure of interactions in the media ecosystem; the extensive literature on ``fake news'' (see~\cite{allcott2017social} discussed elsewhere) and the incentives faced by suppliers of commentary would surely admit the possibility that uncertainty might be introduced or amplified by epidemiological mechanisms, in which case analysis of those mechanisms might yield some insight into whether epidemiological factors (say, the rise of Fox News) have consequences for economic outcomes by changing the degree of economic policy uncertainty.

One way to test for the epidemiological alternative might be consider alternative scenarios for the policies that might be manifested as competing `narratives' about how policymakers will behave; the uncertainty would then be about which narrative would turn out to be correct.  That leads us naturally to our next topic.

% :  housing media sentiment has significant predictive power for future house prices.

\subsubsection{Epidemiology and `Narrative Economics'}\label{narrativeApproach}

Robert Shiller has repeatedly speculated that the driving force in aggregate fluctuations, both for asset markets and for macroeconomies, is the varying prevalence of alternative `narratives' that people believe capture the key `story' of how the economy is working (his earliest statement of this view seems to be \href{https://www.jstor.org/stable/2117915}{\cite{shiller1995conversation}}).

He has returned to this theme more recently, and our opening quote from him makes it clear that he thinks narratives spread by ``going viral.''  See Shiller [\citeyear{shiller2017narrative,shiller_narrative_2019}] for more extended treatments.

There are formidable obstacles to turning Shiller's plausible argument into a quantitative modeling tool.  One is the difficulty of identifying the alternative narratives competing at any time, and reliably measuring their prevalence.
\href{https://github.com/iworld1991/EpiExp/blob/master/Literature/shiller2020popular.pdf}{\cite{shiller2020popular}} made an initial effort at this.  By reading over historical news archives and internet search records, he identified six major economic narratives that have circulated during the economic expansion since 2009, including ``Great Depression,'' ``secular stagnation,'' ``sustainability,'' ``housing bubble,'' ``strong economy,'' and ``save more.''  (See also  \cite{ash2021text} and the references therein.)

\cite{larsen2019business} have recently taken up the challenge of quantifying media narratives, deriving virality indexes, and conducting Granger causality tests to determine the extent to which viral narratives have predictive power for economic outcomes, in the U.S., Japan, and Europe.  The authors do find episodes in which their methodology identifies `narratives' that have `gone viral.'  This is early work, but the authors identify apparent connections between the intensity and valence of discussion of some topics and subsequent economic outcomes.

There is also a rapidly expanding body of work that tries to answer economic questions by analyzing a large volume of textual/conversational information using the developing technique of natural language processing (NLP).  (See \cite{gentzkow2019text} for an overview).

As those tools get more sophisticated, they might become usable for creating more sophisticated and nuanced and reliable methods for tracking the content of narratives in the manner required to turn Shiller's `narrative economics' ideas into practical tools of current analysis.

Separately, it is not beyond imagining that at some point, and to the extent that corporate interests and privacy considerations permit, it might be possible to train AI tools to comb through the vast amount of information contained in social network communications to identify economic narratives, and to measure the ways in which they spread.  Because such a source would have direct measures of the social connections between agents, at that point it might be possible to construct a thoroughly satisfactory epidemiological model of Shiller's narrative theory of economic fluctuations -- and to see how effective it is.  But that date is still some distance in the future.

%More to the definitional point, the process of identifying narratives does not in itself constitute the construction of an epidemiological model.  The narrative is more like the virus itself; a ``full-fledged'' model of epidemiological narrative economics would need to incorporate specifics of transmission channels -- as in the earlier work by \cite{shiller1989survey}.

% \subsubsection{Takeaways}
% The work we have just summarized has some common implications for epidemiological modeling despite the (usual) absence of explicit epidemiological interpretations.  First, ideas differ in their infectiousness depending on many factors including their source and their context.  This suggests that modelers will need to find ways to systematically characterize how such features endogenously affect transmission dynamics.  Second, the structure of social networks -- who you interact with, how frequently, with what intensity -- can affect the process of transmission and potentially the equilibrium outcome.  If Democrats and Republicans do not assort randomly with each other in their social connections or do not accord equal weight to opinions of persons of the opposite party, it is not hard to see how persistent belief differences arise between the two groups, as shown in our Figure~\ref{fig:parker} above showing divergent portfolio choices after the 2016 election.
